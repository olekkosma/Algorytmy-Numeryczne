
\documentclass[12pt]{article}
%	options include 12pt or 11pt or 10pt
%	classes include article, report, book, letter, thesis

\usepackage[a4paper,bindingoffset=0.2in,%
            left=1in,right=1in,top=0.4in,bottom=0.4in,%
            footskip=.2in]{geometry}
            
\usepackage[T1]{fontenc}
\usepackage[polish]{babel}
\usepackage[utf8]{inputenc}
\usepackage{lmodern}
\selectlanguage{polish}
\usepackage{blindtext}
\usepackage{pgfplots}
\usepackage{graphicx}

\title{Algorytmy numeryczne}
\author{Zadanie 1 \\ Aleksander Kosma}
\date{11 Październik 2017}

\begin{document}
\maketitle 

\section{Sumowanie szeregów potęgowych}



Niniejsze opracowanie prezentuje badania i wnioski z prób wyliczenia wartości funkcji arctan.
Odnośnikiem w postaci wyniku były wartości funkcji obliczane przez wbudowaną bibliotekę Math.
Wykorzystałem dwa znane mi sposoby na obliczenie tej wartości. Są to:\\
-zsumowanie elementów wyliczonych z szeregu potęgowego:\\
$$arctg(x) = \sum_{n=0}^{\infty}\frac{(-1)^n}{2n + 1} x^{2n + 1}$$
-zsumowanie elementów wyliczonych na podstawie poprzednika:\\
$$a_{n+1}= a *-\frac{x^2 * (2n + 1)}{2n + 3}$$

Ze względu na charakter obliczeń podzieliłem owe zsumowania na kolejne dwa:\\
-zsumowanie elementów licząc od tyłu\\
-zsumowanie elementów licząc od przodu\\

Pracując na zmiennej typu Double ograniczyłem swoje obliczenia do 15-17 miejsca po przecinku.
Pomimo dość niskiej precyzji obliczeń da się zauważyć pewne zachowania w wynikach. Okazuje się że wyniki
zsumowanych elementów od przodu i od tyłu daje różne rezultaty. Również ze względu na zbieżność szeregu w $[-1,1]$ 
liczyłem tylko dla tego zakresu.

Pierwszy wykres prezentuje procentowe szanse na precyzyjniejszy wynik dla sumowań.
Można zauważyć że bardzo dużą przewagę ma sumowanie od tyłu. Szczególnie w okolicach $[-0.5,0.5]$

\begin{center}
 \makebox[\textwidth]{\includegraphics[width=0.7\paperwidth]{rozklad_procentowy.png}}
\end{center}

Drugi wykres prezentuje bezwzględną różnice miedzy wynikiem biblioteki a sumowaniami. 
Środkowa część wykresu ukazuje wyższość sumowania od tyłu.Jest ono precyzyjniejsze nawet o 2 rzędy wielkości.


\begin{center}
 \makebox[\textwidth]{\includegraphics[width=0.7\paperwidth]{rozklad.png}}
\end{center}

Takie rezultaty wynikają z natury działania zmiennej double. W przypadku sumowania liczby większej
o parę rzędów wielkości do liczby mniejszej, zmienna odrzuca końcowe cyfry mniejszej liczby by pomieścić 
najistotniejsze wartości. Stąd utrata precyzji. W momencie kiedy dwie liczby mają podobną wartość, ich suma potrzebować będzie podobnej precyzji do zapisania wyniku. W konsekwencji nie odrzuci danych.
Podsumowując kwestie precyzji wyniku opierając się o zmienną double, można stwierdzić, że precyzja wyniku wacha się do średnio 14-15 miejsca po przecinku.

By rozwiązać problem który zaistniał w przypadku doubla, trzeba mieć kontrolę nad precyzją wyniku i jego 
ewentualnym zaokrąglaniu. Rozwiązaniem okazuje się klasa BigDecimal.Klasa ta może przechować nieograniczoną 
wartość, ogranicza ją jedynie skala która osiąga wartość 32-bitowego integera, czyli trochę 
ponad dwa miliardy miejsc.



\end{document}